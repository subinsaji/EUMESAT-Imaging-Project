\subsubsection{Main Discussions}
The first averaged image, plotted without the removal of cloud cover shown in Fig.~\ref{fig:av_cloud}, provided a clear outline of the landmasses and enabled easy visual identification of land and sea pixels. It also enabled identification of a subtropical jet stream in the intertropical convergence zone, indicated by the light blue stream of higher density cloud across the image, seen to be travelling across the Atlantic Ocean between Central Africa and South America. This jet stream was still visible despite the pixel averaging due to the near constant cloud cover over the jet stream region throughout the month. From visual inspection of an animation we created from the RGB images in January 2019, this jet stream was seen to be travelling from east to west. Despite the features able to be discerned using this method, due to the inclusion of cloudy pixels in the averaging, other cloud removal techniques are preferred for terrain analysis as cloudy noise was very visible in the image most notably near the southern and northern poles, which presents complications for further analysis techniques of the Earth's surface.
\par
Multiple threshold value selection methods were used to generate masks to remove cloud cover and the plots of these, shown in Fig.~\ref{fig:iterav}, enables the determination of their accuracy via visual comparison with the original RGB image on the left. The second image from the left used the visually selected threshold pixel values for the creation of the cloud mask. When compared to the original image it can be seen that the majority of the cloudy pixels (shown as light areas over the sea and as bright light blue areas over land) were successfully selected as cloud cover. Unfortunately this method was not ideal for detection of darker clouds over the sea (as seen in the upper left hand corner of the image) or at distinguishing between lighter land pixel values and cloud pixel values (a problem that was most noticeable along the coast line and certain regions next to cloud cover), due to the presence of multiple cloud threshold ranges required for thresholding in the IR 1.6 channel. This exists due to the existence of both normal and cold/ice clouds that therefore emitt and reflect less infrared radiation, resulting in darker cloud images.
\par
The adjacent image made use of Otsu's algorithm to calculate the optimum threshold values for the separation of foreground and background pixels. As can be observed in the image, when a single threshold value was applied to both land and sea values the optimum threshold calculated would often group the land and cloud pixel values together as the foreground and the sea pixels as the background. Although this gave a more complete cloud mask over the see compared to threshold determination by inspection, the majority of the land was removed therefore preventing any further terrain analysis to be done on it. 
\par The far right image shows the result of the masked Otsu method that was used to circumvent the previous problem presented by using Otsu's method for threshold value detection. Here it can be seen that that the use of two different threshold values over the sea and land enabled a better detection of cloudy pixels. The percentage of cloud successfully detected appears to be similar to the visually threshold selected image, with the added advantage that very few cloud free pixels (if any) were selected and removed as if they were cloudy.


\subsubsection{Error Discussions}

We noticed that some images in each of the bands were corrupted. These images contain black horizontal or partially white horizontal lines that obscured the southern hemisphere. Meteosat-9 is spin stabilised and scans from east to west and top to bottom. This would help to explain why black or white lines are horizontal. We can only speculate the SIVIRI instrument does occasionally run into a few issues when taking images and it may have decided to neglect an image mid scan for unknown reasons. Despite imaging errors, we continued to use the images to generate a cloud free image. From the results, the  methods is reliable at removing clouds. This may be due to only around 10-20 images are corrupted out of 730 in each band, accounting for only a very small number of total images.
\par
The project also discovered that Otsu's algorithm for thresholding does not work for images that have cloud free days. During cloud free days, the algorithm tries to find the best threshold with pixel values of the land. On average, the pixel values are close to each other so when applying Otsu's thresholding, the images will not look favourable when compared to clouded images. In addition, clouds that are smaller than 5$\times$5$\mathrm{~km}^{2}$ will not be recognised as spatial resolution is simply not high enough to resolve smaller clouds.
\par

\subsection{Taking the project further}

This project only studied three out of twelve spectral bands aboard Meteosat-9. To take this study further, one can apply the same imaging techniques to other spectral bands to see if the results hold the same. Furthermore, it would be interesting to apply these techniques to low Earth orbit (LEO) satellites that are able to image with meter and sub-meter resolution. Using other satellites such as the Landsat can show images in other spectral bands that is useful for studying deciduous and coniferous vegetation, soil and  moisture content. This in turn can lead the way for in-depth normalised difference vegetation index (NDVI) to observe and measure live green vegetation.
\par
This project looked at image segmentation as a method for cloud detection. Due to time constraints, this was not possible. To take the project further, comparisons can be made between different cloud detection and removal algorithms. This can include the use of different ANN architectures. A comparison would yield interesting results.

