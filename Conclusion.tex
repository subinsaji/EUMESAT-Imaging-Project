In this project we have extensively used data from Meteosat-9, a geostationary weather satellite positioned 36,000km above continental Africa. This project has demonstrated that it is possible to re-create an RGB colour image from greyscale images that were taken from a one infrared and two visible channels using the SIVIRI instrument aboard Meteosat-9. This report also demonstrated why investigating cloud coverage is important to solar power generation and the methods that have been used to analyse cloud cover and cloud motion using a variety of techniques such as using artificial neural networks. As a result of introducing our motivation for the report, the project has implemented methods to remove cloud coverage. One such method is was simply averaging over all pixels in 730 images per channel to create a cloud free image of a selected area of Earth. Next, we applied a binary threshold by inspection. Subsequently, the project succeeded in using a more sophisticated method of cloud removal; Otu's method. Moreover, we compared how the cloud coverage changed in the month of January and July. We found that coverage varied significantly due to seasonal effects. Moving on from clouds, we finally investigated the normalised difference vegetation index (NDVI) over a selected area of Earth to find the density of living green vegetation. 



