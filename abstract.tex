


The term ``Remote Sensing’’ was first coined in by Evelyn Pruitt of the United State Office of Naval Research in the 1950s. Today, the term describe the science of observing, measuring and identifying without coming into direct contact with something of interest such as surveying the Earth through satellite imagery at different wavelengths. The technology of remote sensing began with photographic film that were taken from the sky and pointed towards the ground in the 1840s by cameras that were secured to a tethered balloon for topographic mapping. In World War I, cameras were placed on airplanes for reconnaissance. Then, during the the height of Cold War, the U.S. Government developed imaging satellites using specialised photographic film as rocket technology quickly advanced. An example of this is the CORONA photo-satellite resonance system launched in 1960 due to the suspension of U-2 spy plane overflights of the Soviet Union. Figure 1 shows main system of the CORONA after deployment. 
BLAH BLAH BLACH TO  FILL IN LATER